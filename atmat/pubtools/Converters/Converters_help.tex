\documentclass[a4paper,12pt]{report}
\usepackage{booktabs}
\usepackage{siunitx}
\usepackage{color}
\usepackage[table]{xcolor}
\usepackage[cmyk]{xcolor}
\usepackage{amssymb}
\usepackage{amsmath}
\usepackage{longtable}
\usepackage{pdflscape}

\begin{document}
\title{HELP ON: \\ {\color{red!80!black} Converters}}
\author{\color{blue!80!black}atHelp}
\date{\color{green!80!black}\today}
\maketitle
\tableofcontents

\part{Converters}
\chapter{AT2G4BL}

{\bf ATtoG4BL.m}\\
  function [outtext]=ATtoG4BL(P\_0,particle,folder)\\
  tansform AT structure into G4BL input file.\\
 \\
  Simone Maria Liuzzo PhD@ESRF 11-oct-2012\\
\\
\section{Example}
{\bf  Other Files}


{\bf  Other Files}


\chapter{AT2MAD8}

{\bf AT\_2\_mad8.m}\\
  this functions converts the AT lattice AT\_ring in mad8 format.\\
\\
\section{Example}
{\bf  Other Files}


{\bf  Other Files}

 \\
S10.mat \\
mad8elemdef.elem \\
mad8elemdef\_ARCA\_INJ\_def.mad8 \\
mad8elemdef\_ARCB\_INJ\_def.mad8 \\
mad8elemdef\_ARC\_CELL2\_def.mad8 \\
mad8elemdef\_RING\_FF\_def.mad8
\chapter{AT2MADX}

{\bf AT2\_mod\_MADX.m}\\
  transforms the THERING output of AT into a madx readable file. \\
  \\
  tuned for DIAMOND.\\
  \\
  made by Simone Maria Liuzzo 28-6-2011 (PhD)\\
\\

{\bf AT\_2\_madX.m}\\
  this functions converts the AT lattice AT\_ring in mad8 format.\\
\\
\section{Example}
{\bf  Other Files}


{\bf  Other Files}


\chapter{MAD82MADX}

{\bf mad8TOmadx.m}\\
  converts mad8 sequence files to madX\\
 \\
 function [seqfileMADX]=mad8TOmadx(seqfilemad8)\\
 \\
  This procedure reads a saved sequence in\\
  mad8 (SAVE,FILE='seqfilemad8';)\\
  and converts it to madx sequence\\
  every = goes to :=\\
  the order of the declarations is the same in the two files.\\
 \\
  works also with single mad8 files not containing comands, only\\
  definitions.\\
  does not translate call to files since those may change name\\
  \\
  parameters:\\
     - seqdilemad8=name of the mad8 lattice file\\
     - periodname (optional)= name of the period to use in madx (default is the filename)\\
  \\
  Simone Maria Liuzzo PhD@LNF 25-11-2011\\
      update 29-2-2012 : corrected a bug that would not translate correctly\\
      markers, kickers and monitor declared only by element name ("BPM: monitor" would not convet properly)\\
\\
\section{Example}
{\bf  Other Files}


{\bf  Other Files}


\chapter{MADX2AT}

{\bf ParseAtributesMADX\_2\_AT.m}\\
  determines atribute and sets field in sxs{i} structure AT\\
 \\
  created 6-sept-2012\\
\\

{\bf atfrommadx.m}\\
  tansform madX sequence file (savesequence) file into AT lattice structure.\\
 \\
  This procedure reads a saved lattice (sequence in madx) in madX \\
  and converts it to an AT lattice\\
 \\
  (madx comands to save the sequences : \\
  \\
   \_\_\_\_\_\_\_ MADX code \_\_\_\_\_\_\_\_\_\\
   use,period=sequencename1; \\
   use,period=sequencename2; \\
   use,period=sequencename2; \\
   SAVE,FILE='seqfilemadX.seq';\\
   \_\_\_\_\_\_\_\_\_\_\_\_\_\_\_\_\_\_\_\_\_\_\_\_\_\_\_\\
 \\
   seqfilemadX.seq will contain sequencename1 sequencename2 sequencename3 \\
   in the correct format in a single file\\
  \\
  )\\
  \\
  The routine outputs a Matlab macro with all the AT defitions and variables as\\
  in the madX file  \\
 \\
  The order of the declarations is the same in the two files.\\
  declarations that contain other variables are moved to the end. (this may not be enough)\\
 \\
 \\
  Works also with single madX files not containing comands, only\\
  definitions.\\
 \\
  parameters:\\
     - seqfilemadX=name of the mad8 lattice file\\
     - E0  = design energy\\
     - outfilename (default: seqfilemadX\_AT\_LATTICE.mat)\\
     \\
  default pass methods:\\
           quadrupoles : StrMPoleSymplectic4Pass\\
           dipole : BndMPoleSymplectic4Pass\\
           multipole : StrMPoleSymplectic4Pass\\
           sextupole : StrMPoleSymplectic4Pass\\
           thinmultipole : ThinMPolePass\\
           correctors : ThinMPolePass\\
           cavity : DriftPass\\
 \\
\\

{\bf buildATLattice.m}\\
  given a list (cell array) of elements with specified field Spos (center of element (madx default)) in a\\
  vector returns a cell array with elements without Spos field and\\
  appropriate Drifts spaces between. Drifts of the same length have the same name.\\
\\

{\bf reshapeToCellArray.m}\\
  if CEL\_CEL is a cell array of structures and cell arrays it converts it a\\
  cell array of structures.\\
\\
\section{Examples}

{\bf convertMADXtoATExample.m}\\
\\
{\bf  Other Files}

 \\
dbatestsbend2.seq \\
low\_emit\_s10E\_save.seq \\
low\_emit\_s10E\_save\_AT\_LATTICE.mat
{\bf  Other Files}


\chapter{MADX2G4BL}

{\bf madx2g4bl.m}\\
  function [outtext]=madx2g4bl(P\_0,particle,folder)\\
  tansform madx file into G4BL input file.\\
 \\
  Simone Maria Liuzzo PhD@LNF 11-jul-2011\\
\\
\section{Example}
{\bf  Other Files}


{\bf  Other Files}


{\bf  Other Files}

 \\
Converters\_help.aux \\
Converters\_help.log \\
Converters\_help.pdf \\
Converters\_help.synctex.gz \\
Converters\_help.tex \\
Converters\_help.toc

  \end{document}

\documentclass[acus]{article}



\usepackage{booktabs} 
\usepackage{longtable}
\usepackage{subfig}

\usepackage{graphicx}

\begin{document}

\title{AT 2.0 Manual}
\maketitle
\begin{abstract}
We give an overview for the AT 2.0 code and describe how the functions and repository fit into this.
\end{abstract}


\section{Introduction}
AT is a toolbox of functions in Matlab for charged particle beam simulation.
On the one hand, as a toolbox, one should be allowed to use the tools to do what one wants.
On the other, in order to avoid redundancy in work, and facilitate working together via sharing of tools, it is useful to implement some standards.  We describe the tools and some standards and guidelines in AT with the hope that the overall package has some coherence and elegance (speed is also considered at some points where it is critical).

\section{Coordinates}
The 6-d phase space coordinates are as follows
\begin{equation}
\vec Z = \pmatrix{x\cr \frac{p_x}{P_0} \cr y \cr \frac{p_y}{P_0} \cr \delta \cr ct}
\end{equation}
The momenta are defined as
\begin{equation}
x' = \frac{p_x}{P_z} \ \ y' = \frac{p_y}{P_z}
\end{equation}
with $P_z = P_0 (1+\delta)$.


\section{Lattice Creation}
A lattice may be created in a number of ways.  The final result should be a cell array whose elements are the lattice elements.

There are number of different element "classes".  These are
\subsection{Drift}
\begin{itemize}
\item pass method:  DriftPass
\item creation function: ATDRIFT(FAMNAME,LENGTH,PASSMETHOD) creates a drift space element with Class 'Drift'
 FAMNAME		family name
 LENGTH			length [m]
 PASSMETHOD     tracking function, defaults to 'DriftPass'
\end{itemize}

\subsection{Bend}
\begin{itemize}
\item pass methods:  
BndMPoleSymplectic4Pass, BndMPoleSymplectic4RadPass
BndMPoleSymplectic4E2Pass, BndMPoleSymplectic4E2RadPass
BndMPoleSymplectic4FrgFPass, BndMPoleSymplectic4FrgFRadPass
BendLinearPass
\item{element creation functions} 
ATRBEND(FAMNAME,LENGTH,BENDINGANGLE,K,PASSMETHOD)
 	creates a rectangular bending magnet element with class 'Bend'
 		FAMNAME        	family name
 		LENGTH         	length of the arc for an on-energy particle [m]
 		BENDINGANGLE	total bending angle [rad]
 		K				focusing strength, defaults to 0
 		PASSMETHOD      tracking function, defaults to 'BendLinearPass'
 
 ATRBEND(FAMNAME,LENGTH,BENDINGANGLE,K,PASSMETHOD,'FIELDNAME1',VALUE1,...)
    Each pair \{'FIELDNAME',VALUE\} is added to the element

 ATSBEND(FAMNAME,LENGTH,BENDINGANGLE,K,PASSMETHOD)
 	creates a rectangular bending magnet element with class 'Bend'
 		FAMNAME        	family name
 		LENGTH         	length of the arc for an on-energy particle [m]
 		BENDINGANGLE	total bending angle [rad]
 		K				focusing strength, defaults to 0
 		PASSMETHOD      tracking function, defaults to 'BendLinearPass'
 
 ATSBEND(FAMNAME,LENGTH,BENDINGANGLE,K,PASSMETHOD,'FIELDNAME1',VALUE1,...)
    Each pair \{'FIELDNAME',VALUE\} is added to the element
\end{itemize}

\subsection{Quadrupole}
\begin{itemize}
\item{pass methods} StrMPoleSymplectic4Pass, StrMPoleSymplectic4RadPass
QuadLinearPass, QuadMPoleFringePass, QuadMPoleFringeRadPass
ThinMPolePass
\item{element creation function}
ATQUADRUPOLE(FAMNAME,LENGTH,K,PASSMETHOD)
 	creates a quadrupole element with Class 'Quadrupole'
 
 FAMNAME	family name
 LENGTH     length [m]
 K          strength [m-2]
 PASSMETHOD	tracking function, defaults to 'QuadLinearPass'
 
 ATQUADRUPOLE(FAMNAME,LENGTH,K,PASSMETHOD,'FIELDNAME1',VALUE1,...)
    Each pair {'FIELDNAME',VALUE} is added to the element
 \end{itemize}

\subsection{Sextupole}
\begin{itemize}
\item{element creation function}
\item{pass methods}
StrMPoleSymplectic4Pass, StrMPoleSymplectic4RadPass
ThinMPolePass
\subsection{Octupole}
StrMPoleSymplectic4Pass, StrMPoleSymplectic4RadPass
ThinMPolePass
\subsection{Wiggler}
\begin{itemize}
\item{pass methods}
GWigSymplecticPass
\item{element creation function}
atwiggler(fname, Ltot, Lw, Bmax, Nstep, Nmeth, By, Bx, method)
 
  FamName	family name
  Ltot		total length of the wiggle
  Lw		total length of the wiggle
  Bmax	 	peak wiggler field [Tesla]
  Nstep		num of integration steps per period
  Nmeth		symplectic integration method, 2nd or 4th order: 2 or 4
  By		wiggler harmonics for horizontal wigglers
  Bx		wiggler harmonics for vertical wigglers
  method        name of the function to use for tracking
 
  returns a wiggler structure with class 'Wiggler'
\end{itemize}

\subsection{KickMap}
\subsection{RFCavity}
\begin{itemize}
\item{pass methods}
CavityPass
\item{element creation function}
ATRFCAVITY(FAMNAME,LENGTH,VOLTAGE,FREQUENCY,HARMONICNUMBER,ENERGY,PASSMETHOD)
Also, use atsetcavity to set the cavity voltage with and without radiation.
\end{itemize}
\begin{verbatim}
%atsetcavity(ring,rfv, radflag,HarmNumber)
%sets the synchronous phase of the cavity assuming radiation is turned on
%radflag says whether or not we want radiation on, which affects
%synchronous phase.
%also sets the rf voltage and Harmonic number
%also sets the rf frequency.
\end{verbatim}
The synchronous phase with radiation should be
\begin{equation}
\phi_s = sin^{-1}\left(\frac{U_0}{e V_{RF}}\right)
\end{equation}

\subsection{QuantDiff}
\subsection{Monitor}
\subsection{Corrector}
\subsection{Solenoid}
\subsection{Matrix66}
\subsection{RingParam}



The element creation functions are the following:

\subsection{atringparam} 
ATRINGPARAM(rname,E0,per) creates a RingParameter Element which should go at the beginning of the ring\\
 FNAME		Family name which may be used as name of Ring
 Energy     Energy of electrons
 PER        Periodicity of the ring (=1 if ring is already expanded)

\subsection{atmonitor} 
Class: Monitor
\subsection{atmultipole} 
  ATMULTIPOLE(FAMNAME,LENGTH,POLYNOMA,POLYNOMB,PASSMETHOD)
 	creates a multipole element
  	FAMNAME			family name
 	LENGTH			length[m]
 	POLYNOMA        skew [dipole quad sext oct];	 
 	POLYNOMB        normal [dipole quad sext oct]; 
 	PASSMETHOD      tracking function. Defaults to 'StrMPoleSymplectic4Pass'
\subsection{atthinmultipole} 
ATTHINMULTIPOLE(FAMNAME,POLYNOMA,POLYNOMB,PASSMETHOD)
 	creates a thin multipole element with Class 'ThinMultipole'
 
 	FAMNAME			family name
 	POLYNOMA        skew [dipole quad sext oct];	 
 	POLYNOMB        normal [dipole quad sext oct]; 
 	PASSMETHOD      tracking function. Defaults to 'ThinMPolePass'\subsection{atquadrupole}  
 Class: Quadrupole
\subsection{atrbend} 

\subsection{atsbend} 
\subsection{atrfcavity}
 ATRFCAVITY(FAMNAME,LENGTH,VOLTAGE,FREQUENCY,HARMONICNUMBER,ENERGY,PASSMETHOD)
 	creates an rfcavity element with Class 'RFCavity'
 
 		FamName			family name
 		Length			length[m]
 		Voltage			peak voltage (V)
 		Frequency		RF frequency [Hz] 
  		HarmNumber		Harmonic Number
        Energy          Energy in eV          
 		PassMethod		name of the function on disk to use for tracking
\subsection{atsolenoid}
z=solenoid('FAMILYNAME',Length [m],KS,'METHOD')
 	creates a new solenoid element with Class 'Solenoid'
    The structure with field
 	FamName			family name
 	Length			length[m]
 	KS              solenoid strength KS [rad/m]
 	PassMethod     name of the function to use for tracking
 
    function returns assigned address in the FAMLIST that uniquely identifies
    the family
 
    Additional structures being set up (initialized to default values within this routine):   
 	NumIntSteps		Number of integration steps
 	MaxOrder
 	R1					6 x 6 rotation matrix at the entrance
 	R2           		6 x 6 rotation matrix at the entrance
 	T1					6 x 1 translation at entrance 
 	T2					6 x 1 translation at exit
\subsection{atsextupole}  
ATSEXTUPOLE(FAMNAME,LENGTH,S,PASSMETHOD)
 	creates a sextupole element with class 'Sextupole'
 
 FAMNAME		family name
 LENGTH			length [m]
 S				strength [m-2]
 PASSMETHOD     tracking function, defaults to 'StrMPoleSymplectic4Pass'
 \subsection{atwiggler}
 atwiggler(fname, Ltot, Lw, Bmax, Nstep, Nmeth, By, Bx, method)
 
  FamName	family name
  Ltot		total length of the wiggle
  Lw		total length of the wiggle
  Bmax	 	peak wiggler field [Tesla]
  Nstep		num of integration steps per period
  Nmeth		symplectic integration method, 2nd or 4th order: 2 or 4
  By		wiggler harmonics for horizontal wigglers
  Bx		wiggler harmonics for vertical wigglers
  method        name of the function to use for tracking
 
  returns a wiggler structure with class 'Wiggler'
Example: wig = atwiggler('wig1', 1.6, 1.6, .67, 1, 1, [1;0;0;0;0],
[1;0;0;0;0], 'GWigSymplecticPass') 
\subsection{atidtable}  
atidtable(FAMNAME,Nslice,filename,Energy,method)
 
  FamName	family name
  Nslice	number of slices (1 means the wiggler is represented by a
            single kick in the center of the device).
  filename	name of file with wiggler tracking tables.
  Energy    Energy of the machine, needed for scaling
  method    tracking function. Defaults to 'IdTablePass'
 
  The tracking table is described in
  P. Elleaume, "A new approach to the electron beam dynamics in undulators
  and wigglers", EPAC92.
 
  returns assigned structure with class 'KickMap'


\section{Pass Methods}
Each element needs a pass method which takes the phase space coordinate of the electron before the element and returns the phase space coordinate after.  In the absence of radiation, the result will be symplectic.  The pass methods are so-called {\it symplectic integrators}. 

A 4th order symplectic integrator assumes that the Hamiltonian can be written in the form $H = H_1 + H_2$ and that $H_1$ and $H_2$ can be independently solved. 

The Hamiltonian may be written
\begin{equation}
H=1+\delta -(1+hx)\frac{A_s}{B\rho}-(1+hx)\sqrt{(1+\delta)^2-p_x^2-p_y^2}
\end{equation}
and this can be expanded and split.   Now, given a splitting, we define

\begin{equation}
d_1 = d_4 = \frac{1}{2-2^{1/3}} L
\end{equation}

\begin{equation}
d_2 = d_3 = \frac{1-2^{1/3}}{2(2-2^{1/3})} L
\end{equation}

\begin{equation}
k_1 =k_3 = \frac{1}{2-2^{1/3}} L
\end{equation}

\begin{equation}
k_2 = -\frac{2^{1/3}}{2-2^{1/3}} L
\end{equation}

Now, suppose we can solve $H_1$ (drift) and $H_2$ (kick) independently, and we notate

\begin{equation}
e^{:H_1 d:} = D(d)
\end{equation}

\begin{equation}
e^{:H_2:k} = K(k)
\end{equation}
Then the 4th order integrator is

\begin{equation}
D(d_1) K(k_1) D(d_2) K(k_2) D(d_2) K(k_1) D(d_1)
\end{equation}

\subsection{BndMPoleSymplectic4E2Pass}
required arguments:  

    'Length'
    'BendingAngle'
    'EntranceAngle'
    'ExitAngle'
    'PolynomB'
    'MaxOrder'
    'NumIntSteps'


optional arguments:

    'FullGap'
    'FringeInt1'
    'FringeInt2'
    'H1'
    'H2'
    'T1'
    'T2'
    'R1'
    'R2'

\subsection{BndMPoleSymplectic4E2RadPass}
required arguments:
   'Length'
    'BendingAngle'
    'EntranceAngle'
    'ExitAngle'
    'PolynomB'
    'MaxOrder'
    'NumIntSteps'
    'Energy'

optional arguments:
    'FullGap'
    'FringeInt1'
    'FringeInt2'
    'H1'
    'H2'
    'T1'
    'T2'
    'R1'
    'R2'
\subsection{BndMPoleSymplectic4Pass}
required arguments:
   'Length'
    'BendingAngle'
    'EntranceAngle'
    'ExitAngle'
    'PolynomA'
    'PolynomB'
    'MaxOrder'
    'NumIntSteps'

optional arguments:
    'FullGap'
    'FringeInt1'
    'FringeInt2'
    'T1'
    'T2'
    'R1'
    'R2'
\subsection{BndMPoleSymplectic4RadPass}
required arguments:
    'Length'
    'BendingAngle'
    'EntranceAngle'
    'ExitAngle'
    'PolynomA'
    'PolynomB'
    'MaxOrder'
    'NumIntSteps'
    'Energy'

optional arguments:
    'FullGap'
    'FringeInt1'
    'FringeInt2'
    'T1'
    'T2'
    'R1'
    'R2'
\subsection{BendLinearPass}
required arguments:
    'Length'
    'BendingAngle'
    'EntranceAngle'
    'ExitAngle'

optional arguments:
    'K'
    'ByError'
    'FullGap'
    'FringeInt1'
    'FringeInt2'
    'T1'
    'T2'
    'R1'
    'R2'
\subsection{CavityPass}
required arguments:\\
    'Length'
    'Voltage'
    'Energy'
    'Frequency'
optional arguments:\\
    'TimeLag'
\subsection{CorrectorPass}
required arguments:\\
    'Length'
    'KickAngle'
optional arguments:
none
\subsection{DriftPass}
a = 

    'Length'


b = 

     {}
\subsection{QuadLinearPass}
a = 

    'Length'
    'K'


b = 

    'T1'
    'T2'
    'R1'
    'R2'
\subsection{QuadMPoleFringePass}
a = 

    'Length'
    'PolynomA'
    'PolynomB'
    'MaxOrder'
    'NumIntSteps'


b = 

    'T1'
    'T2'
    'R1'
    'R2'
\subsection{StrMPoleSymplectic4Pass}
a = 

    'Length'
    'PolynomA'
    'PolynomB'
    'MaxOrder'
    'NumIntSteps'


b = 

    'T1'
    'T2'
    'R1'
    'R2'
\subsection{StrMPoleSymplectic4RadPass}
a = 

    'Length'
    'PolynomA'
    'PolynomB'
    'MaxOrder'
    'NumIntSteps'
    'Energy'


b = 

    'T1'
    'T2'
    'R1'
    'R2'
\subsection{ThinMPolePass}
required arguments:
    'PolynomA'
    'PolynomB'
    'MaxOrder'

optional arguments:
    'BendingAngle'
    'T1'
    'T2'
    'R1'
    'R2'
\subsection{WiggLinearPass}
required arguments:
    'Length'
    'InvRho'

optional arguments:
    'KxKz'
    'T1'
    'T2'
    'R1'
    'R2'
\subsection{IDTablePass}

required arguments:
    'Length'
    'xkick'
    'ykick'
    'xtable'
    'ytable'
    'Nslice'

optional arguments:
    'xkick1'
    'ykick1'
    'T1'
    'T2'
    'R1'
    'R2'

One possibility is  �(improve this�)
\begin{equation}
H_1 = (1+hx)\frac{p_x^2+p_y^2}{2(1+\delta)}
\end{equation}
\begin{equation}
H_2 = -(1+hx)\frac{A_s}{B\rho}-(1+\delta)hx
\end{equation}
where $h=1/\rho$ with $\rho$ the bending radius of the magnet (for a given energy electron).



\section{Lattice Manipulation}\label{lattice_manip}
There are two kinds of functions for working with lattices.  One type asks a question about certain kinds of elements and
returns a set of indices to those elements.  The function {\it atgetcells} is an important example of this type which asks for field names in the lattice structure and looks for matches.  The output is a set of boolean values corresponding to whether or not each element matches the criterion.  One can also use the Matlab function {\it findcells} in the same way.  The output here is a list of element indices.
(getcellstruct and setcellstruct?)

\subsection{atgetfieldvalues}
 ATGETFIELDVALUES retrieves the field values AT cell array of elements
 
  VALUES = ATGETFIELDVALUES(RING,'field') extracts the values of
  the field 'field' in all the elements of RING
 
  VALUES = ATGETFIELDVALUES(RING,INDEX,'field') extracts the values of
  the field 'field' in the elements of RING selected by INDEX
 
  if RING{I}.FIELD is a numeric scalar
     VALUES is a length(INDEX) x 1 array
  otherwise
     VALUES is a length(INDEX) x 1 cell array
 
 
  More generally ATGETFIELDVALUES(RING,INDEX,subs1,subs2,...) will call
   GETFIELD(RING{I},subs1,subs2,...) for I in INDEX
 
  Examples:
 
  V=ATGETFIELDVALUES(RING,1:10,'PolynomB') is a 10x1  cell array
  such that V{I}=RING{I}.PolynomB for I=1:10
 
  V=ATGETFIELDVALUES(RING(1:10),'PolynomB',{1,2}) is a 10x1 array
  such that V(I)=RING{I},PolynomB(1,2)
\subsection{atsetfieldvalues} 
 ATSETFIELDVALUES sets the field values of MATLAB cell array of structures
 
    Note that the calling syntax must be in the form of assignment:
    RING = ATSETFIELDVALUES(RING,...)
    MATLAB does not modify variables that only appear on the right
    hand side as arguments.
 
 NEWRING=ATSETFIELDVALUES(RING,'field',VALUES)
    In this mode, the function will set values on all the elements of RING
 
 NEWRING=ATSETFIELDVALUES(RING,INDEX,'field',VALUES)
    In this mode, the function will set values on the elements of RING
    specified by INDEX, given as a list of indices or as a logical mask
 
 NEWRING=ATSETFIELDVALUES(RING,'field',VALUESTRUCT)
    In this mode, the function will set values on the elements of RING
    whose family names are given by the field names of VALUESTRUCT
 
 NEWRING=ATSETFIELDVALUES(RING,RINGINDEX,....,VALUESTRUCT)
    As in the previous mode, the function will set values on the elements
    of RING whose family names are given by the field names of VALUESTRUCT.
    But RINGINDEX=atindex(RING) is provided to avoid multiple computations.
 
  Field selection
  ---------------------------------------------------------
  NEWRING = ATSETFIELD(RING,'field',VALUES)
    For each I=1:length(RING), set RING{I}.FIELD=value
 
  NEWRING = ATSETFIELD(RING,'field',{M,N},VALUES)
    For each I=1:length(RING), set RING{I}.FIELD(M,N)=value
 
  More generally,
  NEWRING = ATSETFIELD(RING,subs1,subs2,...,VALUES)
    For each I=1:length(RING), SETFIELD(RING{I},subs1,subs2,...,value)
 
  The last dimension of VALUES must be either length(INDEX) or 1 (the value
  will be repeated for each element). For a vector to be repeated, enclose
  it in a cell array.
 
  Value format
  ---------------------------------------------------------
  Cell array VALUES
  -----------------
  Mx1 cell array : one cell per element
  1x1 cell array : cell 1 is affected to all selected elements
 
  Character array VALUES
  ---------------------
  1xN char array (string) : the string as affected to all selected elements
  MxN char array : one row per element
 
  Numeric array VALUES
  --------------------
  1x1 (scalar) : the value is affected to all selected elements
  Mx1 (column) : one value per element
  MxN (matrix) : one row affected per element. If M==1, the single row
                 is affected to all selected elements
  To assign column vectors to parameters, use a cell array VALUES, with one
  column per cell
 
A lattice manipulation function takes a lattice as an argument and produces a new lattice as a result.
Here are some lattice manipulation functions:

\subsection{Adding errors}
\begin{itemize}
\item atsetshift
\item atsettilt
\item atsetfieldvalues
\item atsplitdrift
\item ataddmpolecomppoly
\item ataddmpoleerrors
\item atloadfielderrs
\end{itemize}


\section{Tracking Particles plus Moments}
The pass methods have two different calling methods.  They may be called directly via the Mex interface (through the MexFunction entry point in the C function), or they may be called indirectly through the function RingPass (through the passFunction entry).  The pass methods should be defined so that calling them with no arguments gives a list of required and optional parameters.

The moment tracking and equilibrium finding occurs via the function OhmiEnvelope().  For this to work requires pass methods that include radiation (this gives a deterministic effect which results in damping and non-symplecticity.  Further, the function findmpoleraddiffmatrix is required to compute the diffusion matrix.

\section{Global parameters and Lattice Functions}
Given the ability to track particles through the lattice, one can compute global beam dynamics parameters and properties that vary around the ring.  
Here are some global parameters.

\subsection{tunes}
\subsection{chromaticity}
\subsection{momentum compaction factor}
\subsection{tune shift with amplitude}

Here are some quantities that vary around the ring and come from particle tracking of one turn:

\subsection{Closed orbit}
\subsection{One turn map matrix}
\subsection{Transfer matrix from one position to another}
\subsection{Twiss Parameters}
\subsection{Dispersion function}


From many turn tracking and determination of stability, one may compute at various points around the ring:

\subsection{dynamic aperture: atdynap}
\subsection{momentum aperture:  atmomentumaperture}


From tracking the diffusion matrix and inclusion of radiation, one can compute:

\subsection{Beam sizes}


\section{Lattice change and optimization from beam parameters}
In addition to the lattice building tools in Section \ref{lattice_manip}, there are also lattice changes one would like to do based on beam parameters.  For example, one would like to change the tunes, chromaticities or beta functions of the lattice.
\begin{itemize}
\item atfittune
\item atfitchrome
\end{itemize}

For general optimization, one may use the atmatch routine.
\subsection{atmatch}
See documentation by S. Liuzzo.

\section{Visualization}
The lattice functions described in the previous section may be plotted, together with a synoptic representation of the lattice.  The function atplot is designed for this purpose.
\begin{figure}[htb]
\centering
\includegraphics*[width=\columnwidth]{images/lat.pdf}
\caption{ESRF lattice from atplot}
\label{fig:esrflat}
\end{figure}



\section{AT within a larger context: Other Codes, Matlab Middle Layer}

\section{Appendix:  List of all functions and their help}
{\bf atlinopt.m}\\
 ATLINOPT               performs linear analysis of the COUPLED lattices\\
 
  LinData $=$ ATLINOPT(RING,DP,REFPTS) is a MATLAB structure array with fields\\
 \\
    ElemIndex   - ordinal position in the RING\\
    SPos        - longitudinal position [m]\\
    ClosedOrbit - closed orbit column vector with\\
                  components x, px, y, py (momentums, NOT angles)\\
    Dispersion  - dispersion orbit position vector with\\
                  components eta\_x, eta\_prime\_x, eta\_y, eta\_prime\_y\\
                  calculated with respect to the closed orbit with\\
                  momentum deviation DP. Only if chromaticity is required.\\
    M44         - 4x4 transfer matrix M from the beginning of RING\\
                  to the entrance of the element for specified DP [2]\\
    A           - 2x2 matrix A in [3]\\
    B           - 2x2 matrix B in [3]\\
    C           - 2x2 matrix C in [3]\\
    gamma       - gamma parameter of the transformation to eigenmodes\\
    mu          - [ mux, muy] horizontal and vertical betatron phase\\
    beta        - [betax, betay] vector\\
    alpha       - [alphax, alphay] vector\\
 \\
    All values are specified at the entrance of each element specified in REFPTS.\\
    REFPTS is an array of increasing indexes that  select elements\\
    from the range 1 to length(LINE)+1.\\
    See further explanation of REFPTS in the ''help'' for FINDSPOS\\
 
  [LinData,NU] $=$ LINOPT() returns a vector of linear tunes
    [nu\_u , nu\_v] for two normal modes of linear motion [1]\\
 
  [LinData,NU, KSI] $=$ LINOPT() returns a vector of chromaticities ksi $=$ d(nu)/(dP/P)\\
    \[ksi u , ksi v\] - derivatives of \[nu u , nu v\]\\
 
  LinData $=$ LINOPT(RING,DP,REFPTS,ORBITIN) does not search for closed orbit.\\
           instead ORBITIN is used\\
 
  Difference with linopt: Fractional tunes 0<$=$tune<1\\
                  Dispersion output (if chromaticity is required)\\
                  Alpha output\\
                  Phase advance output\\
                  Option to skip closed orbit search\\
 
  See also ATREADBETA ATX ATMODUL FINDSPOS TWISSRING TUNECHROM\\
 
    \[1\] D.Edwars,L.Teng IEEE Trans.Nucl.Sci. NS-20, No.3, p.885-888, 1973\\
    \[2\] E.Courant, H.Snyder\\
    \[3\] D.Sagan, D.Rubin Phys.Rev.Spec.Top.-Accelerators and beams, vol.2 (1999)\\


{\bf atradon.m}\\
 ATRADON               switches RF and radiation on\\
 \\
 RING2$=$ATRADON(RING,CAVIPASS,BENDPASS,QUADPASS)\\
 \\
 RING:          initial AT structure\\
 CAVIPASS:     pass method for cavities (default ThinCavityPass)\\
 BENDPASS:     pass method for cavities (default BndMPoleSymplectic4RadPass)\\
 QUADPASS:     pass method for cavities (default: nochange)\\
 \\
 \[RING2,RADINDEX,CAVINDEX\]$=$ATRADON(...) returns the index of radiative elements\\
                           and cavities\\
\\

{\bf atx.m}\\
 ATX                    computes and displays global information\\
 \\
 BEAMDATA$=$ATX(RING,DPP,REFPTS)\\
 \\
 RING:          AT structure\\
 DPP:          relative energy deviation (default: 0)\\
 REFPTS:    Index of elements (default: 1:length(ring))\\
 \\
 BEAMDATA is a MATLAB structure array with fields\\
 \\
  From atlinopt:\\
 \\
    ElemIndex   - ordinal position in the RING\\
    SPos        - longitudinal position [m]\\
    ClosedOrbit - closed orbit column vector with\\
                  components x, px, y, py (momentums, NOT angles)\\
    Dispersion  - dispersion orbit position vector with\\
                  components eta\_x, eta\_prime\_x, eta\_y, eta\_prime\_y\\
                  calculated with respect to the closed orbit with\\
                  momentum deviation DP\\
    M44         - 4x4 transfer matrix M from the beginning of RING\\
                  to the entrance of the element for specified DP [2]\\
    A           - 2x2 matrix A in [3]\\
    B           - 2x2 matrix B in [3]\\
    C           - 2x2 matrix C in [3]\\
    gamma       - gamma parameter of the transformation to eigenmodes\\
    mu          - [ mux, muy] horizontal and vertical betatron phase\\
    beta        - [betax, betay] vector\\
    alpha       - [alphax, alphay] vector\\
 \\
  From ohmienvelope:\\
 \\
    beam66      - 6x6 equilibrium beam matrix\\
    emit66      - 6x6 emittance projections on x and y + energy spread\\
    beam44      - intersection of beam66 for dpp$=$0\\
    emit44      - emittances of the projections of beam44 on x and y\\
    modemit     - [emitA emitB] emittance of modes A and B (should be constant)\\
 \\
 \[BEAMDATA,PARAMS\]$=$ATX(...)  Returns also a structure PM with fields\\
    ll          - Circumference\\
    alpha       - momentum compaction factor\\
    nuh         - Tunes\\
    nuv\\
    fulltunes\\
    fractunes\\
    espread     - Energy spread\\
    blength     - Bunch length\\
    modemittance - Eigen emittances\\
 \\
  See also: ATREADBETA ATLINOPT OHMIENVELOPE ATMODUL\\
\\

\begin{thebibliography}{2}

\bibitem{KMW}
A. Terebilo \emph{Accelerator Toolbox for Matlab}, SLAC-PUB 8732 (May 2001)

\end{thebibliography}

\end{document}

